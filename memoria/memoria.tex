\documentclass[11pt]{article}
\usepackage[utf8]{inputenc}
\usepackage{graphicx}


\title{\textbf{Deep Learning for Detecting Objects in Images with Robots using TensorFlowjs and Websim/Kibotics}}
\author{Author: Jorge Cruz de la Haza\\
		Tutor: Dr. Jose Maria Cañas Plaza\\
		Co-tutor: Prf. Julio Manuel Vega Pérez}

\date{Septiembre 2020}
\begin{document}

\maketitle
\clearpage
\tableofcontents
\clearpage

\section{Introduction}

Summary of the project

This is how to write cursive and footnotes \emph{cursive}\footnote{https://launchpad.net/rubber/} \& \emph{other}\footnote{http://www.phys.psu.edu/{\textasciitilde}collins/software/latexmk-jcc/}.

All inside the {} get the cursive mode \emph{(View $\rightarrow$ Page layout in preview)}.

\subsection{Robots}
\subsection{Deep Learning}
\subsubsection{Machine Learning on Computer Vision}
\subsubsection{Neural Networks}
\subsubsection{Convolutional Neural Networks (CNNs)}
\subsection{Teaching with robotics: JdeRobot}

\section{Objectives}
\subsection{Objectives}
\subsection{Methodology}
\subsection{Requirements}

\section{Infrastructure}
\subsection{Websim}
\subsubsection{RobotAPI, Worker and Miniproxy}
\subsection{Javascript}
\subsection{OpenCVjs Library}
\subsection{TensorFlowjs}
\subsubsection{MNIST}
\subsubsection{COCO-SSD}


\section{Autonomous robot in simulated city}
\subsection{PiBot}
\subsection{Simulated city}
\subsubsection{Blender}
\subsubsection{Gltf models}
\subsubsection{Gltf animated models}

\subsection{Signals Detector}
\subsection{Follow Road}

\section{Conclusions}

\section{Bibliography}

\end{document}
